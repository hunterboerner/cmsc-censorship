\section{Introduction to Filesharing Protocols}\label{sec:intro}

% Adversaries on the Internet may attempt to prevent access to various
% information. As a result of these censorship efforts developers have made a
% variety of systems for circumventing said blocks. The largest variety of efforts
% have been targeted at sharing files with fewer attempts at providing general
% access to the internet over \textsc{http}/\textsc{https}.

% We will examine various protocols for filesharing and their susceptibility to
% censorship.

We begin our overview of filesharing on the Internet with an examination of
networked filesharing protocols and Usenet. These methods predate
much of a concern for secure communications, in fact, encryption technologies
were once subject to both legal and export restrictions.
We will defer our discussion of censorship to a more detailed discussion of
censorship techniques in section \S
\ref{censorship}.

\subsection{Networked Filesharing Protocols}

There are three main operating systems in use around the world: Windows,
\textsc{gnu}/Linux, and macOS. Correspondingly, there are three main filesharing
protocols---\textsc{smb}, \textsc{nfs}, and \textsc{afs}. The differences
between the protocols isn't very important. What matters is that these are
commonly used internally within a network and not for the public to access.

% \subsubsection{Apple File Protocol (\textsc{afp})}

% AFP is a legacy protocol for file sharing in older macOS environments. It is
% able to preserve macOS-specific metadata, which enables plug-and-play file
% Some of its key features include\cite{eriksenCOMPARISONNFSSAMBA}:
% \begin{itemize}[nosep]
%       \item A stateful connection shares for macOS users.
%       \item Metadata caching tailored to small-size macOS files Support for
%             Kerberos single-sign-on and standard \textsc{posix} permissions on
%             shares for \textsc{afp}~3.1+
%       \item Fast transmission speed for macOS workflows that require Mac-specific
%             metadata
%       \item Minimal support in non-Apple operating systems
% \end{itemize}
% The \textsc{afp} protocol has been replaced with \textsc{smb}, even in macOS
% environments.

% \subsubsection{Network File System (\textsc{nfs})}

% The NFS protocol is a fairly filesharing protocol described in RFC
% 1094\cite{NFSNetworkFile1989,eriksenCOMPARISONNFSSAMBA}. It applies primarily to Unix and Linux installations, particularly in servers. It was designed to provide a lightweight and stateless mechanism through which Unix hosts might mount remote directories transparently. Its key features include:
% \begin{itemize}[nosep]
%       \item Statelessness for version 2 or 3, with optional statefulness in
%             version 4;
%       \item Employs flush-on-close strategy for caching;
%       \item Relies on trusting clients unless DES or Kerberos is configured
%             explicitly;
%       \item Provides outstanding performance on Linux/Unix, especially for
%             frequent small read and writes;
%       \item Achieves optimal performance when all clients are Unix/Linux, but
%             still functions on Windows with extra client software.
% \end{itemize}
% The NFS protocol is still prevalent on Linux/Unix platforms today.
% \subsubsection{Server Message Block (\textsc{smb})}

% The SMB protocol is a widely-used file sharing protocol primarily designed for
% Windows environments. Its key features include\cite{eriksenCOMPARISONNFSSAMBA}:
% \begin{itemize}[nosep]
%       \item Statefulness and the using of opportunistic locks to boost speed;
%       \item Read-ahead and write-behind functions together with opportunistic
%             locks to help clients cache aggressively;
%       \item Security measures such as Kerberos, NTLMv2, message signing and
%             optional encryption;
%       \item Optimal performance on Windows and newer macOS with SMB 3 support;
%       \item Excellent compatibility as the protocol is built into most modern
%             platforms including Windows, macOS, and Linux.
% \end{itemize}
% The SMB protocol has successfully replaced AFP even in pure macOS environments.
% It has therefore become the default choice for modern networks with mixed devices.

How would one host files that they need to share externally? They can either run
an HTTP(S) server such as Apache or Nginx and also enable the \texttt{autoindex}
mode. You may have seen this sort of webpage while browsing kernel files.

Another option is to run an FTP server. FTP stands for File Transfer Protocol
and as such shares files. FTP doesn't come with encryption or authentication
built in though it does offer tooling that is more advanced that the HTTP(S)
route.

A common threat to filesharing is the Digital Millennium Copyright Act
\cite{DigitalMillenniumCopyright2025} which allows for \emph{takedown requests}
to be sent to sites that are hosting copyrighted content without permission to
be sent to sites that are hosting copyrighted content without permission. If you
get a DMCA request (usually sent to your Internet Services Provider, or ISP) you
have to take down the content or have your internet connection shut off. The
best way to avoid getting taken down with a server of copyrighted files is to
stay away from indexers that would put you under the spotlight.


\subsection{Usenet}\label{usenet}

A system that transcends the usual client-server model of downloading is Usenet.
Usenet emerged in 1980 at Duke University, where it initially served as a
bulletin board system but later evolved into a global network of distributed
discussion forums.\cite{HowUseUsenet} It organized discussions into thousands of
newsgroups covering different topics, where users would post messages in
threads. By 1999, the size of Usenet had already reached approximately 20,000
newsgroups, making it one of the most diverse platforms on the Internet.

Paul Ohm described a framework of four stages for internet ``spaces''. Stage I
is ``before any new spaces have been created.'' Stage II includes spinning off
of spaces (e.g., a Usenet archive which is itself a separate space from the
Usenet server which expires messages). Stage III is more mature with fewer
spin-offs. Finally, Stage IV ``marks stability and possibly decline.'' As per
Ohm's framework, Usenet is an exemplar "Stage II" Internet space which remains
vibrant despite having moved beyond the initial creation phase---it still
accommodates frequent creation of alternative spaces, evidenced by the
community's strong self-governance. This self-governance includes removing
content that is against ``netiquette'' (discussed below) or illegal (a DMCA
takedown). Usenet files that are subject to a DMCA takedown can be downloaded
before the DMCA takedown request is processed. Then the file can be
redistributed on bittorrent (\S \ref{bittorrent}) which allows files to be
redistributed with few limitations.

\subsubsection{The Architecture of Usenet} The design of
Usenet was unprecedented at the time -- instead of storing data centrally at
servers, Usenet messages are designed to be stored in a decentralized way in
that messages are propagated from server to server throughout the network via a
``flooding routing algorithm'', where each news server maintains connections to
one or more neighbor servers via dedicated ``news feeds.'' \cite{HowUseUsenet}
The Network News Transfer Protocol (\textsc{nntp}) is thereby invented: when a
new message arrives, each server compares the message's unique Message-ID header
against local database to prevent duplicate storage, and then forwards novel
messages to all neighboring servers. As a result of Usenet's decentralized
nature, "no one party or organization controls or can control Usenet" since
``every administrator controls [their] own site''.

\subsubsection{Self-Governance: the Netiquette}

In order to regulate online behavior in Usenet forums, informal rules, later
termed ``netiquette'' as a portmanteau of ``net'' and ``etiquette'', came into
existence among early Usenet users.\cite{HowUseUsenet} The extent of netiquette
covered guidelines for commercial posts, cross-posting limits, and subject
conventions. A set of escalating responses were applied as conflict resolution
mechanisms, ranging from polite education and "frequently asked questions"
(\textsc{faq}) files to more aggressive technical countermeasures such as
banning. Earliest quality control measures also arose, including but not limited
to requiring descriptive subject lines for easier filtering, creating moderated
newsgroups where posts are reviewed pre-publication, and developing keyword
systems.

\subsubsection{Case Study: The Cancel Wars}
The Usenet was equipped with a built-in ``cancel-message'' feature, which was
originally designed to enable users to delete their own posts. However, the tool
was soon weaponized when users discovered that forging addresses allowed them to
cancel anyone's messages. This feature was utilized by anti-spam activists --
who began with good intentions---to empower automated ``cancelbots'' to detect
and remove commercial spams, while others revenge by creating ``resurrector
bots'' that automatically repost cancelled messages. The conflict between two
parties eventually escalated to include ``Usenet Death Penalities'' (cancelling
all posts from entire ISPs) and sophisticated filtering systems such as
``NoCeM'' that hide rather than delete undesirable messages.

% \subsection{Mesh Networks}

If you have bought a home WiFi router in recent years you might have seen
``mesh'' networks. These are WiFi devices that relay messages in the zone of a
single access point (AP) to extend the range of your connection. Similarly, mesh
networks can be used for communication.
% \cite{kamaliAnixAnonymousBlackoutResistant}
The beauty of mesh topology lies in
its self-healing properties---when one node fails, the network automatically
reroutes traffic through alternative paths, making it inherently resilient.
% \cite{akyildizWirelessMeshNetworks2005,cilfoneWirelessMeshNetworking2019}
We can easily extend this usage of text-based communication to sharing files.
In fact, there has been considerable research in this area, particularly around
peer-to-peer file sharing architectures that leverage the multi-hop nature of
mesh networks.
% \cite{alasaadPeerpeerFileSharing2009,yangMeshFSDistributedFile2017}.
These
systems can distribute content efficiently across mesh nodes, with some
implementations using network coding techniques to optimize data transmission.
% \cite{kimbaditadamouCooperativeFileSharing2009a}
What makes mesh networks
particularly interesting for censorship resistance is their decentralized
nature---there is no single point of control that can be easily targeted or shut
down.
% \cite{ivanovicChartingCensorshipResilience2024a}.

\subsection{Bittorrent}
\label{bittorrent}

Bittorrent (BT) is one of the most popular ways of sharing files online. The
idea behind BT is that the \emph{parts} of a file which are
\emph{least-available} are the ones that need to be propagated first to improve the performance of the network. In the early days of BT \emph{trackers} were used to track which clients had different parts of the files. The downside of this is that trackers could be taken down! One notable examples is The Pirate Bay which is frequently taken down in one nation and then pops back up somewhere else.

An alternative to trackers was needed and this is where the Distributed Hash Table comes in. Rather than going to a server to find the peers, the DHT is a global hash table that stores mapping from torrent hashes to the clients that can be connected.

\subsubsection{Legal Issues}

We discussed in the section on Usenet (\S \ref{usenet}) that DMCA takedown
requests can remove illegal files off of the network. In the case of Bittorrent,
the takedowns are issues at \emph{you}, the user. The exception to this rule is
if you're using some sorts of Virtual Private Networks (VPNs, below).

\subsection{Virtual Private Networks (VPNs)}\label{vpn}

A Virtual Private Network (VPN) is a network that operates on top of the
underlying network infrastructure which presents a view of disparate computers
connected under a single infrastructure. For example, if I have a server, say,
\texttt{alice.example.com}, hosted on AWS, and my PC at home,
\texttt{bob.example.com}, at home, then I can create a VPN that I can connect to
which will make it look like the servers are on the same network.

The usage of VPNs in the case of using Bittorrent is a bit different. Here, the
goal is to pass all of our network traffic through a third party so that we
don't leak our IP address to copyright holders which will then try to get our
internet service shut off. VPNs operate through weird legal mechanisms and often
\emph{don't keep logs} so even when they do get a takedown request they don't
know you're the one they need to give it to.

\subsection{Shadow Libraries}\label{shadow}

Shadow Libraries are centralized webpages\footnote{Examples: LibGen, SciHub, or,
      now, Anna's Archive, \url{https://annas-archive.org/}} that provide stolen
access to manuscripts and scanned media. These sites then may redistribute
the files through \emph{mirrors}, Tor (\S \ref{tor}), I2P (\S \ref{i2p}),
or IPFS (\S \ref{ipfs}). Shadow libraries operate on a wide array of
domains as they are frequently taken down.

\subsection{Tor}\label{tor}

The Onion Router (Tor, \cite{TorProjectPrivacy}) is a novel system for providing
anonymous access to the Internet via \emph{circuits} of guard, relay, and exit
nodes. When you run Tor, your computer picks the three aforementioned nodes and
then encrypts your message in layers (like an onion) so that the message can be
transmitted across the circuit such that each node only knows the previous and
next stops.

% TODO: \cite{iszaevichDistributedDetectionTor2019} discusses Tor in Mexico.
% \cite{hollerCaseStudyDDoS2024} discusses DDOS attacks on Tor relays.


Tor uses centralized Directory Authorities \cite{TorProjectDirectory} to create
the list of relays, a potential attack vector. Tor uses fixed circuits for
bidirectional communication.\cite{AnalyzingTrendsTor} Tor relays have one of
four specialized roles: Guard, Middle, Exit, and Bridge. Unlike I2P, clients are
not also routers.

Tor supports \emph{pluggable transports} \label{pluggable_transports} which is a
mechanism that allows Tor traffic to look like a different protocol, such as
HTTP or HTTPS.

\subsection{I2P}\label{i2p}

I2P, the Invisible Internet Project \cite{I2PAnonymousNetwork} has similar
goals to Tor but is primarily concerned with \emph{eepsites}, which are the
equivalent fo Tor's onion services. I2P uses a DHT for routers unlike Tor's
centralized Directory Authorities.\cite{NetworkDatabaseI2P} Despite this, I2P
has centralization when ``bootstrapping'' using the reseed servers.

I2P uses uni-directional packet switching \cite{I2PComparedTor} whereas Tor uses
bi-directional circuits. Furthermore, I2P uses \emph{garlic routing} which is
very similar to onion routing but where related messages going to the next hop
are bundled together.

\subsection{IPFS}\label{ipfs}

IPFS \cite{IPFSBuildingBlocks} is a distributed filesystem. It has mechanisms
for DMCA takedowns and is generally considered easier to censor that Bittorrent,
and certainly easier than I2P or Tor. You can download desktop software to add
files to IPFS and you can also \emph{pin} other files to keep them on your
system. If you don't pin a file its local storage is treated as a cache
\cite{HostSinglepageWebsite} and can be removed to make space for other files.

% \subsection{Legally Hardened System}

% There is a paper that describes a \textsc{p2p} system which doesn't require a
% file to be stored (even encrypted) on a
% node.\cite{CensorshipresistantAnonymousP2P} This provides the advantage that if
% a censor knows the fingerprint of a file then they will no be able to block the
% file on this network as only the receiver is able to identify the file being
% transmitted.


% % In the \textsc{ipfs} there is (or was) a mechanism for censoring content by
% % messing with the Distributed Hash
% % Table.\cite{sridharContentCensorshipInterPlanetary2023}

% \subsection{Motivations for Stealing}
% In a study conducted in Singapore, the authors found that users of \textsc{p2p}
% file sharing did so for the following reasons:\cite{MoreJustFree}
% \begin{enumerate}[nosep]
%       \item Public methods are too slow;
%       \item Hard to find or censored content;
%       \item Sampling content;
%       \item Because it's free.
% \end{enumerate}

\section{Censorship Methods}\label{censorship}

In this section, we outline the various technical methods that can be used to
censor filesharing protocols, analyzing how these various filesharing protocols
are disrupted across different layers of the stack. We focus on real-world
examples such as Iran, and the Great Firewall of China, and evaluate the
effectiveness and limitations to each approach.

At the IP layer censorship can be carried out through rerouting or dropping IPv4/IPv6 packets.
\cite{wendzelSurveyInternetCensorship2025} If we look back to the previously
mentioned network file protocols (NFS, AFP, and SMB), bittorrent, Usenet, and
FTP then we can imagine how a censor would be able to detect these (unencrypted)
protocols and start dropping the IP packets. I2P also suffers a similar issue of
revealing its protocol.  Tor, in contrast, attempts to avoid detection through obfuscation, 
including pluggable transports (\S \ref{pluggable_transports}). Packet fingerprinting techniques 
can be applied to spot non-standard protocols or encrypted payloads that deviate from normative 
traffic patterns.

Recent reports highlight Iran's deployment of a \emph{protocol filter} which blocks any
protocol besides DNS, HTTP, and HTTPS.
\cite{bockDetectingEvadingCensorshipDepth} In this setup once a disallowed
protocol is detected all of the client to server data is dropped. With TCP this
means that (acknowledgement) ACK packets cannot be sent which effectively kills the connection.
\cite[\S 4.1]{bockDetectingEvadingCensorshipDepth} Strangely, ports not used for
DNS, HTTP, and HTTPS are \emph{not} filtered. As Iran’s system does not block traffic based on port 
numbers, this opens possible avenues for circumvention through crafted packet sequences.
Since the filter only looks at the first two packets, it is theoretically possible to 
circumvent the filter by sending two special first packets, initially ignored by the 
server. Strategies include sending a sequence of empty packets or multiple FIN/ACK messages 
to manipulate the connection state machine on both client and server sides.

BGP based censorship is too coarse grained to be an effective file sharing
censorship technique. While taking an entire country offline will prevent
downloading files, it also prevents any and all other traffic.\cite[p.\@
      5]{wendzelSurveyInternetCensorship2025}

Censors often interfere with DNS in order to prevent access to filsharing resouces. A
common attack is where censors monitor DNS queries and replies with spoofed or poisoned 
responses faster than the legitimate server. This approach is used to block access to shadow libraries, 
which frequently change domains in response. As one domain is seized or blocked, another is launched. 
For instance SciHub and LibGen often operate across multiple top-level domains to evade these 
DNS-level takedowns. 

More advanced DNS attacks include DNS injection and recursive poisoning. These attacks can disrupt 
not only access to filesharing domains but also block access to circumvention tools such as proxy 
lists and key exchange servers. The combination of DNS filtering along with IP filtering can be devastatingly 
effective against naive users.

Some censorship goes beyond DNS and IP manipulation, through attacking the routing infrastructure. Turkey
has previously committed this instance of censorship by conducting BGP hijacks to redirect traffic from 
blocked services. Turkish Authorities, in 2014, initiated a BGP hijack in which they rerouted traffic 
from these blocked services, to their own DNS resolvers. This attack poisoned upstream routes at the ISP level, 
thus ensuring that users traffic would still be intercepted and misdirected even if they manually 
set alternative DNS servers. BGP based censorship like this is particularly destructive as it not only 
affects domestic users but can also spill over into neighboring networks, creating collateral damage \cite{chung2017dnssec}. 

The transport layer (TCP, UDP) is another place where censorship can be
implemented.\cite[p.\@ 5]{wendzelSurveyInternetCensorship2025}

DNS can also be targeted even if the adversary can only monitor the DNS
connection. In this situation the adversary can respond faster than the
authoritative DNS server with a bogus DNS response.\cite[p.\@
      6]{wendzelSurveyInternetCensorship2025} Shadow libraries (\S \ref{shadow})
frequently lose their domain host and can have their DNS records manipulated.
This is why there are so many Top-Level Domains hosting SciHub---one is taken
down, another one goes up.

HTTP, when unencrypted, provides an excellent surface to enable filtering. This
is because the HTTP payload provides ample context to determine if a request
should be censored. If a users searches for a dissident in China and the request
is readable by the censors, then the censor can just filter the result. This is
how Baidu works with its integration with the CCP. The NSA has used a
\emph{man-on-the-side} attack that is reminiscent of the DNA fronting hijack
whereby an adversary responds faster than the authentic site, causing the user
to download malware instead of the site content. The legitimate content is then
seen as duplicate messages and is dropped by the network card.

Other unencrypted application protocols suffer the same issues. In mainland
China, many Western applications have been cloned and replaced with surveillance
and censorship functionality to achieve the goals of the CCP.

Although we mentioned HTTPS has a better attack surface, it is not immune. HTTPS
uses the Server-Name Indication to convey the server domain in cleartext so that
web servers can determine which certificate to use to decrypt the message. There
is a workaround to this: Encrypted SNI (ESNI). \cite[p.\@
      8]{wendzelSurveyInternetCensorship2025} In this setup the client gets a DNS TXT
record to determine what key to use to sign the ESNI. Then the events proceed as
normal except the SNI is encrypted.

Since VPNs usually have multiple users at a time, the endpoint of a VPN can be
blocked by an ISP or nation-state which will disallow access to anyone using
that VPN.\cite[p.\@ 8]{wendzelSurveyInternetCensorship2025} This can easily be
done with IP blocking or port filtering but also can be achieved with Deep
Packet Inspection, although this is much more costly.

Tor might be vulnerable to \emph{timing attacks} if the adversary has access to
enough Tor nodes.\cite[p.\@ 9]{wendzelSurveyInternetCensorship2025}

In China, where the Tor Directory Nodes are blocked, the alternative methods of getting on to the Tor network represent another attack vector.

There is another approach that we have only briefly mentioned: malware. If you
can infect a computer with malware you can do almost anything you want to the
user.

A \emph{smear attack} is where content forums are manipulated to make the site
less attractive. If your favorite forum is suddenly overrun with propaganda you
may be less inclined to visit it.

In Iran's DNS censorship they made an error with the filtering code
\cite{bockDetectingEvadingCensorshipDepth} such that the first DNS over TCP
packet is never blocked but subsequent messages are. Similarly, the HTTPS filter
just ensures TLS is used, it doesn't actually block anything! The authors
prevent a few ways of circumventing the filter. For the second one where two
\textsc{fin} packets are sent before the connection starts, the researchers note
``We do not understand why this strategy works[.]'' Another strategy is to send
nine \textsc{ack}s or any nine non-data-carrying packets.
