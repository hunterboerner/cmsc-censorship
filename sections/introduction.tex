\section{Introduction}\label{sec:intro}

Adversaries on the Internet may attempt to prevent access to various
information. As a result of these censorship efforts developers have made a
variety of systems for circumventing said blocks. The largest variety of efforts
have been targeted at sharing files with fewer attempts at providing general
access to the internet over \textsc{http}/\textsc{https}.

We will examine various protocols for filesharing and their susceptibility to
censorship.

\subsection{Networked Filesharing Protocols}

There are three main operating systems in use around the world: Windows,
\textsc{gnu}/Linux, and macOS. Correspondingly, there are three main filesharing
protocols---\textsc{smb}, \textsc{nfs}, and \textsc{afs}. In this section we will
compare them.

\subsubsection{Apple File Protocol (\textsc{afp})}

AFP is a legacy protocol for file sharing in older macOS environments. It is
able to preserve macOS-specific metadata, which enables plug-and-play file
Some of its key features include\cite{eriksenCOMPARISONNFSSAMBA}:
\begin{itemize}[nosep]
    \item A stateful connection shares for macOS users.
    \item Metadata caching tailored to small-size macOS files Support for
          Kerberos single-sign-on and standard \textsc{posix} permissions on
          shares for \textsc{afp}~3.1+
    \item Fast transmission speed for macOS workflows that require Mac-specific
          metadata
    \item Minimal support in non-Apple operating systems
\end{itemize}
The \textsc{afp} protocol has been replaced with \textsc{smb}, even in macOS
environments.

\subsubsection{Network File System (\textsc{nfs})}

The NFS protocol is a fairly filesharing protocol described in RFC
1094\cite{NFSNetworkFile1989,eriksenCOMPARISONNFSSAMBA}. It applies primarily to Unix and Linux installations, particularly in servers. It was designed to provide a lightweight and stateless mechanism through which Unix hosts might mount remote directories transparently. Its key features include:
\begin{itemize}[nosep]
    \item Statelessness for version 2 or 3, with optional statefulness in
          version 4;
    \item Employs flush-on-close strategy for caching;
    \item Relies on trusting clients unless DES or Kerberos is configured
          explicitly;
    \item Provides outstanding performance on Linux/Unix, especially for
          frequent small read and writes;
    \item Achieves optimal performance when all clients are Unix/Linux, but
          still functions on Windows with extra client software.
\end{itemize}
The NFS protocol is still prevalent on Linux/Unix platforms today.
\subsubsection{Server Message Block (\textsc{smb})}

The SMB protocol is a widely-used file sharing protocol primarily designed for
Windows environments. Its key features include\cite{eriksenCOMPARISONNFSSAMBA}:
\begin{itemize}[nosep]
    \item Statefulness and the using of opportunistic locks to boost speed;
    \item Read-ahead and write-behind functions together with opportunistic
          locks to help clients cache aggressively;
    \item Security measures such as Kerberos, NTLMv2, message signing and
          optional encryption;
    \item Optimal performance on Windows and newer macOS with SMB 3 support;
    \item Excellent compatibility as the protocol is built into most modern
          platforms including Windows, macOS, and Linux.
\end{itemize}
The SMB protocol has successfully replaced AFP even in pure macOS environments.
It has therefore become the default choice for modern networks with mixed devices.
\subsection{Usenet}
\subsubsection{Historical Context\cite{HowUseUsenet}}
Usenet emerged in 1980 at Duke University, where it initially served as a
bulletin board system but later evolved into a global network of distributed
discussion forums. It organized discussions into thousands of newsgroups
covering different topics, where users would post messages in threads. By 1999,
the size of Usenet had already reached approximately 20,000 newsgroups, making
it one of the most diverse platforms on the Internet.

\subsubsection{The Architecture of Usenet\cite{HowUseUsenet}}
The design of Usenet was unprecedented at the time -- instead of storing data
centrally at servers, Usenet messages are designed to be stored in a
decentralized way in that messages are propagated from server to server
throughout the network via a "flooding routing algorithm", where each news
server maintains connections to one or more neighbor servers via dedicated
``news feeds.'' The Network News Transfer Protocol (\textsc{nntp}) is thereby
invented: when a new message arrives, each server compares the message's unique
Message-ID header against local database to prevent duplicate storage, and then
forwards novel messages to all neighboring servers. As a result of Usenet's
decentralized nature, "no one party or organization controls or can control
Usenet" since ``every administrator controls [their] own site''.

\subsubsection{Self-Governance: the Netiquette\cite{HowUseUsenet}}

In order to regulate online behavior in Usenet forums, informal rules, later
termed ``netiquette'' as a portmanteau of ``net'' and ``etiquette'', came into
existence among early Usenet users. The extent of netiquette covered guidelines
for commercial posts, cross-posting limits, and subject conventions. A set of
escalating responses were applied as conflict resolution mechanisms, ranging
from polite education and "frequently asked questions" (\textsc{faq}) files to
more aggressive technical countermeasures such as banning. Earliest quality
control measures also arose, including but not limited to requiring descriptive
subject lines for easier filtering, creating moderated newsgroups where posts
are reviewed pre-publication, and developing keyword systems.

\subsubsection{Case Study: The Cancel Wars}


\cite{HowUseUsenet,ohmRegulatingInternetUsenet1998}

\subsection{Bittorrent}

\subsection{Virtual Private Networks (VPNs)}

\subsection{Shadow Libraries}

\subsection{Tor}
\subsection{I2P}

\subsection{IPFS}
