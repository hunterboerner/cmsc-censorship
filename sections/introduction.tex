\section{Introduction}\label{sec:intro}

Adversaries on the Internet may attempt to prevent access to various
information. As a result of these censorship efforts developers have made a
variety of systems for circumventing said blocks. The largest variety of efforts
have been targeted at sharing files with fewer attempts at providing general
access to the internet over \textsc{http}/\textsc{https}.

We will examine various protocols for filesharing and their susceptibility to
censorship.

\subsection{Networked Filesharing Protocols}

There are three main operating systems in use around the world: Windows,
\textsc{gnu}/Linux, and macOS. Correspondingly, there are three main filesharing
protocols---\textsc{smb}, \textsc{nfs}, and \textsc{afs}. The differences
between the protocols isn't very important. What matters is that these are
commonly used internally within a network and not for the public to access.

% \subsubsection{Apple File Protocol (\textsc{afp})}

% AFP is a legacy protocol for file sharing in older macOS environments. It is
% able to preserve macOS-specific metadata, which enables plug-and-play file
% Some of its key features include\cite{eriksenCOMPARISONNFSSAMBA}:
% \begin{itemize}[nosep]
%       \item A stateful connection shares for macOS users.
%       \item Metadata caching tailored to small-size macOS files Support for
%             Kerberos single-sign-on and standard \textsc{posix} permissions on
%             shares for \textsc{afp}~3.1+
%       \item Fast transmission speed for macOS workflows that require Mac-specific
%             metadata
%       \item Minimal support in non-Apple operating systems
% \end{itemize}
% The \textsc{afp} protocol has been replaced with \textsc{smb}, even in macOS
% environments.

% \subsubsection{Network File System (\textsc{nfs})}

% The NFS protocol is a fairly filesharing protocol described in RFC
% 1094\cite{NFSNetworkFile1989,eriksenCOMPARISONNFSSAMBA}. It applies primarily to Unix and Linux installations, particularly in servers. It was designed to provide a lightweight and stateless mechanism through which Unix hosts might mount remote directories transparently. Its key features include:
% \begin{itemize}[nosep]
%       \item Statelessness for version 2 or 3, with optional statefulness in
%             version 4;
%       \item Employs flush-on-close strategy for caching;
%       \item Relies on trusting clients unless DES or Kerberos is configured
%             explicitly;
%       \item Provides outstanding performance on Linux/Unix, especially for
%             frequent small read and writes;
%       \item Achieves optimal performance when all clients are Unix/Linux, but
%             still functions on Windows with extra client software.
% \end{itemize}
% The NFS protocol is still prevalent on Linux/Unix platforms today.
% \subsubsection{Server Message Block (\textsc{smb})}

% The SMB protocol is a widely-used file sharing protocol primarily designed for
% Windows environments. Its key features include\cite{eriksenCOMPARISONNFSSAMBA}:
% \begin{itemize}[nosep]
%       \item Statefulness and the using of opportunistic locks to boost speed;
%       \item Read-ahead and write-behind functions together with opportunistic
%             locks to help clients cache aggressively;
%       \item Security measures such as Kerberos, NTLMv2, message signing and
%             optional encryption;
%       \item Optimal performance on Windows and newer macOS with SMB 3 support;
%       \item Excellent compatibility as the protocol is built into most modern
%             platforms including Windows, macOS, and Linux.
% \end{itemize}
% The SMB protocol has successfully replaced AFP even in pure macOS environments.
% It has therefore become the default choice for modern networks with mixed devices.

How would one host files that they need to share externally? They can either run
an HTTP(S) server such as Apache or Nginx and also enable the \texttt{autoindex}
mode. You may have seen this sort of webpage while browsing kernel files.

Another option is to run an FTP server. FTP stands for File Transfer Protocol
and as such shares files. FTP doesn't come with encryption or authentication
built in though it does offer tooling that is more advanced that the HTTP(S)
route.

A common threat to filesharing is the Digital Millennium Copyright Act
(\cite{DigitalMillenniumCopyright2025}) which allows for \emph{takedown
      requests} to be sent to sites that are hosting copyrighted content without
permission to be sent to sites that are hosting copyrighted content without
permission. If you get a DMCA request (usually sent to your Internet Services
Provider, or ISP) you have to take down the content or have your internet
connection shut off. The best way to avoid getting taken down with a server of
copyrighted files is to stay away from indexers that would put you under the
spotlight.


\subsection{Usenet}

A system that transcends the usual client-server model of downloading is Usenet.
Usenet emerged in 1980 at Duke University, where it initially served as a
bulletin board system but later evolved into a global network of distributed
discussion forums.\cite{HowUseUsenet} It organized discussions into thousands of
newsgroups covering different topics, where users would post messages in
threads. By 1999, the size of Usenet had already reached approximately 20,000
newsgroups, making it one of the most diverse platforms on the Internet.

Paul Ohm described a framework of four stages for internet ``spaces''. Stage I
is ``before any new spaces have been created.'' Stage II includes spinning off
of spaces (e.g., a Usenet archive which is itself a separate space from the
Usenet server which expires messages). Stage III is more mature with fewer
spin-offs. Finally, Stage IV ``marks stability and possibly decline.'' As per
Ohm's framework, Usenet is an exemplar "Stage II" Internet space which remains
vibrant despite having moved beyond the initial creation phase---it still
accommodates frequent creation of alternative spaces, evidenced by the
community's strong self-governance.\cite{ohmRegulatingInternetUsenet1998} This
self-governance includes removing content that is against ``netiquette''
(discussed below) or illegal (a DMCA takedown). Usenet files that are subject to
a DMCA takedown can be downloaded before the DMCA takedown request is processed.
Then the file can be redistributed on bittorrent (\ref{bittorrent}) which allows
files to be redistributed with few limitations.

\subsubsection{The Architecture of Usenet} The design of
Usenet was unprecedented at the time -- instead of storing data centrally at
servers, Usenet messages are designed to be stored in a decentralized way in
that messages are propagated from server to server throughout the network via a
``flooding routing algorithm'', where each news server maintains connections to
one or more neighbor servers via dedicated ``news feeds.'' \cite{HowUseUsenet}
The Network News Transfer Protocol (\textsc{nntp}) is thereby invented: when a
new message arrives, each server compares the message's unique Message-ID header
against local database to prevent duplicate storage, and then forwards novel
messages to all neighboring servers. As a result of Usenet's decentralized
nature, "no one party or organization controls or can control Usenet" since
``every administrator controls [their] own site''.

\subsubsection{Self-Governance: the Netiquette}

In order to regulate online behavior in Usenet forums, informal rules, later
termed ``netiquette'' as a portmanteau of ``net'' and ``etiquette'', came into
existence among early Usenet users.\cite{HowUseUsenet} The extent of netiquette
covered guidelines for commercial posts, cross-posting limits, and subject
conventions. A set of escalating responses were applied as conflict resolution
mechanisms, ranging from polite education and "frequently asked questions"
(\textsc{faq}) files to more aggressive technical countermeasures such as
banning. Earliest quality control measures also arose, including but not limited
to requiring descriptive subject lines for easier filtering, creating moderated
newsgroups where posts are reviewed pre-publication, and developing keyword
systems.

\subsubsection{Case Study: The Cancel Wars}
The Usenet was equipped with a built-in ``cancel-message'' feature, which was
originally designed to enable users to delete their own posts. However, the tool
was soon weaponized when users discovered that forging addresses allowed them to
cancel anyone's messages. This feature was utilized by anti-spam activists --
who began with good intentions -- to empower automated ``cancelbots'' to detect
and remove commercial spams, while others revenge by creating ``resurrector
bots'' that automatically repost cancelled messages. The conflict between two
parties eventually escalated to include ``Usenet Death Penalities'' (cancelling
all posts from entire ISPs) and sophisticated filtering systems such as
``NoCeM'' that hide rather than delete undesirable messages.



\cite{HowUseUsenet,ohmRegulatingInternetUsenet1998}

\subsection{Mesh Networks}

\cite{kamaliAnixAnonymousBlackoutResistant}

\subsection{Bittorrent} \label{bittorrent}

\subsection{Virtual Private Networks (VPNs)}

\subsection{Shadow Libraries}

\subsection{Tor}

TODO: \cite{iszaevichDistributedDetectionTor2019} discusses Tor in Mexico.
\cite{hollerCaseStudyDDoS2024} discusses DDOS attacks on Tor relays.

Tor uses centralized Directory Authorities \cite{TorProjectDirectory} to create
the list of relays.

Tor uses fixed circuits for bidirectional
communication.\cite{AnalyzingTrendsTor} Tor relays have one of four specialized
roles: Guard, Middle, Exit, and Bridge. Unlike I2P, clients are not also
routers.

\subsection{I2P}

I2P uses a DHT for routers unlike Tor's centralized Directory
Authorities.\cite{NetworkDatabaseI2P} Despite this, I2P has centralization when
``bootstrapping'' using the reseed servers.

I2P uses uni-directional packet switching \cite{I2PComparedTor}.

\subsection{IPFS}
