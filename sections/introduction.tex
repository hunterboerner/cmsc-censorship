\section{Introduction}\label{sec:intro}

Adversaries on the Internet may attempt to prevent access to various
information. As a result of these censorship efforts developers have made a
variety of systems for circumventing said blocks. The largest variety of efforts
have been targeted at sharing files with fewer attempts at providing general
access to the internet over \textsc{HTTP}/\textsc{HTTPS}.

We will examine various protocols for filesharing and their susceptibility to
censorship.

\subsection{Networked Filesharing Protocols}

There are three main operating systems in use around the world: Windows,
GNU/Linux, and macOS. Correspondingly, there are three main filesharing
protocols---SMB, NFS, and AFP. In this section we will compare them.

\subsubsection{Apple File Protocol (AFP)}

AFP is a legacy protocol for file sharing in older macOS environments. It is
able to preserve macOS-specific metadata, which enables plug-and-play file
Some of its key features include:
\begin{itemize}[nosep]
  \item A stateful connection shares for macOS users.
  \item Metadata caching tailored to small-size macOS files
        Support for Kerberos single-sign-on and standard POSIX permissions on shares for AFP 3.1+
  \item Fast transmission speed for macOS workflows that require Mac-specific metadata
  \item Minimal support in non-Apple operating systems
\end{itemize}
It has been replaced with SMB.

\subsubsection{Network File System (NFS)}

The NFS protocol is a fairly filesharing protocol described in RFC
1094\cite{NFSNetworkFile1989}. It applies primarily to Unix and Linux installations, particularly in servers. It was designed to provide a lightweight and stateless mechanism through which Unix hosts might mount remote directories transparently. Its key features include:
\begin{itemize}[nosep]
  \item Statelessness for version 2 or 3, with optional statefulness in version 4;
  \item Employs flush-on-close strategy for caching;
  \item Relies on trusting clients unless DES or Kerberos is configured explicitly;
  \item
\end{itemize}

\subsubsection{Server Message Block (SMB)}


\subsection{Usenet}


\subsection{Bittorrent}

\subsection{Virtual Private Networks (VPNs)}

\subsection{Shadow Libraries}

\subsection{Tor}
\subsection{I2P}

\subsection{IPFS}
