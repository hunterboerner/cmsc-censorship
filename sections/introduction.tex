\section{Introduction}\label{sec:intro}

Adversaries on the Internet may attempt to prevent access to various
information. As a result of these censorship efforts developers have made a
variety of systems for circumventing said blocks. The largest variety of efforts
have been targeted at sharing files with fewer attempts at providing general
access to the internet over HTTP/HTTPS.

We will examine various protocols for filesharing and their susceptibility to
censorship.

\subsection{Networked Filesharing Protocols}

There are three main operating systems in use around the world: Windows,
GNU/Linux, and macOS. Correspondingly, there are three main filesharing
protocols---SMB, NFS, and AFP. In this section we will compare them.

\subsubsection{Apple File Protocol (AFP)}

AFP is a legacy protocol for file sharing on macOS. It has been replaced by SMB.

\subsubsection{Network File System (NFS)}

The NFS protocol is a fairly filesharing protocol described in RFC
1094\cite{NFSNetworkFile1989}.


\subsubsection{Server Message Block (SMB)}


\subsection{Usenet}


\subsection{Bittorrent}

\subsection{Virtual Private Networks (VPNs)}

\subsection{Shadow Libraries}

\subsection{Tor}
\subsection{I2P}

\subsection{IPFS}
